\vspace*{0.5cm}\noindent\begin{tabularx}{\textwidth}{@{}>{\tb}l@{} @{\hspace{1cm}}>{\tsmall}X@{}}
หัวข้อวิทยานิพนธ์		   & \begin{tabular}[t]{@{}X@{}}\printTitleInThai\end{tabular}\vspace{11pt}\\
ผู้เขียน				 & \printNamePrefixInThai\printAuthorInThai\vspace{11pt}\\
ปริญญา				  & \printDegreeInThai (\printProgramInThai)\vspace{11pt}\\
อาจารย์ที่ปรึกษา	 	  & \printAdvisorInThai
\end{tabularx}

\begin{ThaiAbstract}
% This paper presents an investigation into the potential of echo state networks (ESN) as a model for reducing resource usage in Earthquake Early Warning (EEW) systems. EEW systems are crucial for mitigating the impact of earthquakes, but they can be resource-intensive for training and prediction, both in terms of computation and time consumption. We propose using ESN, a type of reservoir computing, as an alternative to conventional deep learning EEW models such as convolutional neural networks (CNN). The results of our experimentation show that the ESN model can achieve similar performance to CNN models while using fewer resources. Our findings demonstrate the potential of ESN as a more efficient approach for EEW systems and suggest that further research in this area is warranted.

วิทยานิพนธ์นี้นำเสนอการประยุกต์ใช้และหาศักยภาพของโครงข่ายข่ายสถานะสะท้อน เพื่อลดการใช้ทรัพยากรในระบบแจ้งเตือนแผ่นดินไหวล่วงหน้า ระบบนี้เป็นสิ่งสำคัญสำหรับลดความเสียหายจากเหตุการณ์แผ่นดินไหว แต่ระบบอาจต้องการทรัพยากรในการคำนวณและเวลามหาศาลเพื่อการเรียนรู้ของโมเดลและการทำนาย งานวิจัยนี้เสนอการใช้ โครงข่ายข่ายสถานะสะท้อนซึ่งเป็นการคำนวณแบบที่จัดเก็บน้ำชนิดหนึ่ง เพื่อเป็นทางเลือกแทนโครงข่ายประสาทเทียมแบบสังวัตนาการ ผลการทดลองของเราแสดงว่า โครงข่ายข่ายสถานะสะท้อนสามารถมีประสิทธิภาพเทียบเท่าโครงข่ายประสาทเทียมแบบสังวัตนาการในขณะที่ใช้ทรพยากรน้อยลง การค้นพบของเราแสดงให้เห็นถึงศักยภาพของ โครงข่ายข่ายสถานะสะท้อนเป็นวิธีการที่ทำให้ระบบ ระบบแจ้งเตือนแผ่นดินไหวล่วงหน้ามีประสิทธิภาพมากขึ้นและแนะนำให้มีการวิจัยเพิ่มเติมเพื่อเป็นการรับรองผลลัพธ์
\end{ThaiAbstract}