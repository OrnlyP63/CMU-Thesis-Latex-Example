% \begin{thebibliography}{99}
% \bibitem{b1} G. O. Young, ``Synthetic structure of industrial plastics,'' in \emph{Plastics,} 2\textsuperscript{nd} ed., vol. 3, J. Peters, Ed. New York, NY, USA: McGraw-Hill, 1964, pp. 15--64.
% % Reference 1
% \bibitem{ref-journal1}
% Y.-J. Chiang, T.-L. Chin, and D.-Y. Chen, "Neural Network Based Strong Motion Prediction for On-Site Earthquake Early Warning," \emph{Sensors,} vol. 22, 2022, pp. 704.

% % Reference 2
% \bibitem{ref-journal2}
% D. Jozinovic, A. Lomax, I. Stajduhar, and A. Michelini, "Rapid prediction of earthquake ground shaking intensity using raw waveform data and a convolutional neural network," \emph{Geophys. J. Int.,} vol. 222, 2020, pp. 1379-1389.

% % Reference 3
% \bibitem{ref-journal3}
% Z. Li, M.-A. Meier, E. Hauksson, Z. Zhan, and J. Andrews, "Machine learning seismic wave discrimination: Application to earthquake early warning," \emph{Geophysical Research Letters,} vol. 45, 2018, pp. 4773-4779.

% % Reference 4
% \bibitem{ref-journal4}
% S.-M. Mousavi and G.-C. Beroza, "A machine-learning approach for earthquake magnitude estimation," \emph{Geophysical Research Letters,} vol. 47, 2020.

% % Reference 4-1
% \bibitem{ref-journal4-1}
% S.-M. Mousavi, W.-L. Ellsworth, W. Zhu, L.-Y. Chuang, and G.-C. Beroza, "Earthquake transformer—an attentive deep learning model for simultaneous earthquake detection and phase picking," \emph{Nature Communications,} vol. 11, no. 1, 2020.

% % Reference 8
% \bibitem{ref-journal8}
% D. Yang and K. Yang, "Multi-step prediction of strong earthquake ground motions and seismic responses of SDOF systems based on EMD-ELM method," \emph{Soil Dynamics and Earthquake Engineering,} vol. 85, 2016, pp. 117-129.

% % Reference 9
% \bibitem{ref-journal9}
% S.-M. Mousavi, Y. Sheng, Y. Zhu, and G.-C. Beroza, "STanford EArthquake Dataset (STEAD): A Global Data Set of Seismic Signals for AI," \emph{IEEE Access,} vol. 7, 2019, pp. 179464-179476.

% % Reference 10
% \bibitem{ref-journal10}
% K.-M. Asim, F. Martı´nez-A´-lvarez, A. Basit, and T. Iqbal, "Earthquake magnitude prediction in Hindukush region using machine learning techniques," \emph{Nat Hazards,} vol. 85, 2017, pp. 471-486.

% % Reference 12
% \bibitem{ref-journal12}
% C. Lian, Z. Zeng, W. Yao, and H. Tang, "Displacement prediction model of landslide based on a modified ensemble empirical mode decomposition and extreme learning machine," \emph{Nat Hazards,} vol. 66, 2013, pp. 759-771.

% % Reference 13
% \bibitem{ref-journal13}
% M.-A. Salam, R. Ibrahim, and D.-S. Abdelminaam, "Earthquake Prediction using Hybrid Machine Learning Techniques," \emph{International Journal of Advanced Computer Science and Applications,} vol. 9, no. 4, 2018

% % Reference 18
% \bibitem{ref-journal18}
% T. Perol, M. Gharbi, and M. Denolle, "Convolutional neural network for earthquake detection and location," {\em Science Advances}, vol. 12, 2021.

% % costEEW01
% \bibitem{ref-costEEW01}
% A. Wu, J. Lee, I. Khan, and Y. W. Kwon, "CrowdQuake+: Data-driven Earthquake Early Warning via IoT and Deep Learning," {\em IEEE International Conference on Big Data (Big Data)}, pp. 2068-2075, 2021.

% % costEEW02
% \bibitem{ref-costEEW02}
% Z. Li, M. A. Meier, E. Hauksson, Z. Zhan, and J. Andrews, "Machine Learning Seismic Wave Discrimination: Application to Earthquake Early Warning," {\em Geophysical Research Letters}, vol. 45, pp. 4773-4779, 2018.

% \bibitem{ref-EEW01}
% P. Gasparini, G. Manfredi, and J. Zschau, {\em Earthquake Early Warning Systems}, 1st ed. Springer, United States, 2007, pp. 1--4.

% \bibitem{ref-EEW02}
% R. Allen and D. Melgar, "Earthquake Early Warning: Advances, Scientific Challenges, and Societal Needs," {\em Annu. Rev. Earth Planet. Sci.}, vol. 47, pp. 361-88, 2019.

% \bibitem{ref-EEW03}
% N. C. Hsiao, Y. M. Wu, T. C. Shin, L. Zhao, and T. L. Teng, "Development of earthquake early warning system in Taiwan," {\em Geophysical Research Letters}, vol. 36, 2009.

% \bibitem{ref-RC}
% M. Cucchi, S. Abreu, G. Ciccone, D. Brunner, and H. Kleemann, "Hands-on reservoir computing: a tutorial for practical implementation," {\em Neuromorph. Comput. Eng.}, vol. 2, 2022.

% \bibitem{ref-RC02}
% F. Bianchi, S. Scardapane, S. Lokse, and R. Jenssen, "Reservoir Computing Approaches for Representation and Classification of Multivariate Time Series," {\em IEEE Transactions on Neural Networks and Learning Systems}, vol. 32, pp. 2169-2179, 2021.

% \bibitem{ref-RC03}
% H. Jaeger, "Adaptive nonlinear system identification with echo state networks," in {\em Proc. NIPS}, 2002.
% \end{thebibliography}
\bibliographystyle{ieeetr}
\bibliography{source}
