\vspace*{0.4cm}\noindent\begin{tabularx}{\textwidth}{@{}>{\bfseries}l@{} @{\hspace{1cm}}X@{}}
Thesis Title	& \begin{tabular}[t]{@{}X@{}}\printTitleForAbstract\end{tabular}\vspace{11pt}\\
Author			& \printNamePrefix\printAuthor\vspace{11pt}\\
Degree			& \printDegree~(\printProgram)\vspace{11pt}\\
Advisor	& \printAdvisor
\end{tabularx}

\begin{Abstract}
This paper presents an investigation into the potential of echo state networks (ESN) as a model for reducing resource usage in Earthquake Early Warning (EEW) systems. The EEW systems are crucial for mitigating the impact of earthquakes, but they can be resource-intensive for training and prediction, both in terms of computation and time consumption. We propose using ESN, a type of reservoir computing, as an alternative to conventional deep learning EEW models such as convolutional neural networks (CNN). The results of our experimentation show that the ESN model can achieve similar performance to CNN models while using fewer resources. Our findings demonstrate the potential of ESN as a more efficient approach for EEW systems and suggest that further research in this area is warranted.
\end{Abstract}